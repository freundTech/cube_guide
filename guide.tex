\documentclass[a4paper,11p]{article}
\usepackage{geometry}
\usepackage[english, german]{babel}
\usepackage[utf8]{inputenc}
\usepackage{hyperref}
\usepackage{syntonly}
\usepackage{graphicx}
\usepackage{etoolbox}
%\syntaxonly
\geometry{margin=1in}

\newcommand{\bigimage}[1]{\includegraphics[width=3cm]{images/#1}}
\newcommand{\smallimage}[1]{\includegraphics[width=2cm]{images/#1}}
\newcommand{\alg}[1]{
	\renewcommand*{\do}[1]{\includegraphics[width=1cm]{images/##1}\hspace{0.2cm}}
	\docsvlist{#1}
}

\begin{document}
\pagestyle{plain}
\date{}
\title{Rubik's Cube Algorithmen}
\maketitle
\section{Erste Ebene}
\subsection{Kreuz}
\bigimage{cross}
\begin{enumerate}
	\item \smallimage{cross_case1} 
	\alg{F,F}
	\item \smallimage{cross_case2}
	\alg{D,M,D',M'}
	\item \smallimage{cross_case3}
	\alg{R',D',R}, dann Fall 1 oder 2
\end{enumerate}

\subsection{Ecken der ersten Ebene}
\bigimage{first_layer}
\begin{enumerate}
	\item \smallimage{first_layer_case1}
	\alg{D',R',D,R}
	\item \smallimage{first_layer_case2}
	\alg{D,F,D',F'}
	\item \smallimage{first_layer_case3}
	\alg{D',R',D,D,R}, dann Fall 1
\end{enumerate}

\section{Zweite Ebene}
\bigimage{second_layer}
Hier Würfel umdrehen
\begin{enumerate}
	\item \smallimage{second_layer_case1}
	\alg{U,R,U,R',U',F',U',F}
	\item \smallimage{second_layer_case2}
	\alg{U',L',U',L,U,F,U,F'}
\end{enumerate}

\section{Letzte Ebene}
\bigimage{solved}
\subsection{Ecken tauschen}
\smallimage{last_layer_swap_corners}
\alg{F,R,U',R',U',R,U,R',F',U,U}
\subsection{Ecken drehen}
\smallimage{last_layer_rotate_corners}
\alg{R',D,R,D'} bis Ecke stimmt. Anschließend mit \alg{U} nächste Ecke an die Stelle drehen und Algorithmus weiter ausführen
\subsection{Kanten tauschen}
\smallimage{last_layer_swap_edges}
\alg{M',U,M,U,U,M',U,M}
\subsection{Kanten drehen}
\smallimage{last_layer_rotate_edges}
\alg{R,E} bis Kante stimmt. Anschließend mit \alg{U} nächste Kante an die Stelle drehen und Algorithmus weiter ausführen
\end{document}